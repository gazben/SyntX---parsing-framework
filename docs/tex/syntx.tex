% The MIT License (MIT)
% 
% Copyright (c) 2013, Gergely Nagy
% 
% Permission is hereby granted, free of charge, to any person obtaining a copy
% of this software and associated documentation files (the "Software"), to deal
% in the Software without restriction, including without limitation the rights
% to use, copy, modify, merge, publish, distribute, sublicense, and/or sell
% copies of the Software, and to permit persons to whom the Software is
% furnished to do so, subject to the following conditions:
% 
% The above copyright notice and this permission notice shall be included in
% all copies or substantial portions of the Software.
% 
% THE SOFTWARE IS PROVIDED "AS IS", WITHOUT WARRANTY OF ANY KIND, EXPRESS OR
% IMPLIED, INCLUDING BUT NOT LIMITED TO THE WARRANTIES OF MERCHANTABILITY,
% FITNESS FOR A PARTICULAR PURPOSE AND NONINFRINGEMENT. IN NO EVENT SHALL THE
% AUTHORS OR COPYRIGHT HOLDERS BE LIABLE FOR ANY CLAIM, DAMAGES OR OTHER
% LIABILITY, WHETHER IN AN ACTION OF CONTRACT, TORT OR OTHERWISE, ARISING FROM,
% OUT OF OR IN CONNECTION WITH THE SOFTWARE OR THE USE OR OTHER DEALINGS IN
% THE SOFTWARE.
\documentclass[12pt]{article}
\usepackage[paper=a4paper, margin=2.5cm]{geometry}
\usepackage{fancyhdr}
\usepackage[utf8]{inputenc}
\usepackage[english]{babel}
\usepackage[pdfborder={0 0 0}]{hyperref}
\usepackage{setspace}
\usepackage[T1]{fontenc}
\usepackage[lighttt]{lmodern}	%needed to have bold teletype (to have bold keywords in lstlisting)
\usepackage{hfoldsty}			%has to go after \usepackage{lmodern} to take effect
\usepackage{listings}

\pagestyle{fancy}
\fancyhf{}
\renewcommand{\headrulewidth}{0pt}
\renewcommand{\footrulewidth}{0pt}
\fancyhf[FL]{\scriptsize{\textit{The SyntX Parser Library}}}
\fancyhf[FC]{\scriptsize{\textit{\textbf{\thepage}}}}
\fancyhf[FR]{\scriptsize{\textit{\today}}}

%\lstset{basicstyle=\scriptsize\ttfamily, keywordstyle=\bfseries\ttfamily}
\lstset{basicstyle=\scriptsize\ttfamily, stringstyle=\ttfamily,
keywordstyle=\bfseries\ttfamily,aboveskip=10pt,
belowskip=10pt, tabsize=2, xleftmargin=5pt, captionpos=b, 
literate={á}{{\'a}}1 {é}{{\'e}}1 {ó}{{\'o}}1 {ö}{{\"o}}1 {ú}{{\'u}}1 {ü}{{\"u}}1}

\newcommand{\usec}[2]{\section*{#1}\label{sec:#2}\addcontentsline{toc}{section}{#1}}
\newcommand{\usubsec}[2]{\subsection*{#1}\label{subsec:#2}\addcontentsline{toc}{subsection}{#1}}

\begin{document}
\sloppy

\vspace*{2em}
\centerline{\Huge\textbf{The SyntX Parser Library}}
\vspace{5em}

\begin{center}
	\begin{minipage}[h]{0.85\linewidth}
		\itshape
		The SyntX parser library is a set of C++ classes that enables users to create a parser for an
		LL(k) grammar by giving the production rules with an EBNF-like notation as C++ expressions. There is
		thus no need for a precompiler, the parser can be an integral part of a C++ project.
	\end{minipage}
\end{center}

\vspace{4.5em}

\usec{Recursive descent parsing}{rdp}
The SyntX library is based upon the theory of recursive descent parsers (RDPs). In order to gain a deep and confident
understanding of the library, one has to be familiar with the rudiments of recursive descent parsing.

This section provides a brief introduction with a very simple example. It also introduces some of the types
and notations that are used in the source code of the library as well.

\usubsec{The Extended Backus-Naur Form}{ebnf}
EBNF\footnote{Extended Backus-Naur Form --
\href{http://en.wikipedia.org/wiki/Ebnf}{en.wikipedia.org/wiki/Ebnf}} is a widespread notation that can be
used to express context-free grammars. It is the extension of the Backus-Naur Form (BNF) and actually several
EBNF styles exist. This document uses the one defined by W3C\footnote{World Wide Web Consortium --
\href{http://www.w3.org}{www.w3.org}}.

The "Hello World" grammar that is used to introduce parsing describes a mathematical expression containg
numbers, the four basic operations and an arbitrary depth of parentheses. The grammar below is even simpler
than that, it allows only addition, so the way precedence can be handled is not shown. It allows an infinite
depth of grouping using parentheses however. The ease with which recursive structures can be handled by
EBNF grammars and RDPs is a very important feature and is often exploited.

\begin{center}
	\begin{minipage}[h]{0.5\textwidth}
		\begin{lstlisting}[breaklines=true]
addition = addend ('+' addend)*
addend = [0-9] | expression
expression = '(' addition ')'
		\end{lstlisting}
	\end{minipage}
\end{center}

The production rule named \texttt{addition} is the \emph{start rule}. The text that we're trying to analyze
should conform to this rule as a whole. The details can be investigated by following the references to other
rules.

The \texttt{addition} rule states that an addition is made up of an addend that can be followed by an
arbitrary number of addends each preceded by a '+' operator. The \texttt{*} operator in EBNF allows zero or
more occurence of a rule (or a sequence of rules). This means that an addition may consist of a single addend
as well, which is OK: a number can be thought of as a mathematical expression.

If we wanted to have at least two addends seperated by a '+' operator, than EBNF's operator \texttt{+} should
be used, which allows one or more occurences.

An addend in our simple grammar can be either a single digit (given here by the shorthand used to describe
character sequences in EBNF) or an expression. The \texttt{|} operator stands for \emph{alternation}.

The interesting part lays in the \texttt{expression} rule. An expression is an addition enclosed by
parantheses. This is how the grammar becomes recursive and how an expression of infinite complexity can be
described by a handful of production rules.

Let's see some examples! The simplest possible expression is a single digit:

\begin{center}
	\begin{minipage}[h]{0.1\textwidth}
		\begin{lstlisting}[breaklines=true]
8
		\end{lstlisting}
	\end{minipage}
\end{center}

The start rule is where the matching starts, so we assume that the our text is an \texttt{addition}. An addition starts
with an \texttt{addend}. An \texttt{addend} can be a digit so the text indeed starts with an addend. Thus the
first part of the \texttt{addition} rule matched, we may move on to the rest of the text. Next we're looking
for a '+' character which we don't find. Luckily this part of the rule is optional, so we get to the end of
the rule having matched the entire text. This is a successful match.

Let's go for something more complex next:

\begin{center}
	\begin{minipage}[h]{0.2\textwidth}
		\begin{lstlisting}[breaklines=true]
2 + 3 + (4 + 8)
		\end{lstlisting}
	\end{minipage}
\end{center}

The \texttt{addition} rule will find two digits with a '+' character in between. It could already stop there
as we have a sum which complies to the rules, but we haven't reached the end of the text and actually the
rule allows more than two addends. So the parsing continues, we find a second '+' character and then an
opening parenthesis. This is OK, as the \texttt{addend} rule doesn't only match digits, \texttt{expression}s
are also allowed.

So next we're trying to match the \texttt{expression} rule which contains an \texttt{addition} between
parentheses. That's exactly what we have here so, again, we have a successful match. Please note that the
\texttt{addition} inside the \texttt{expression} can again turn into an \texttt{expression} and then into an
\texttt{addition} -- this is exactly how the arbitrary depth of the expression is analyzed.

\usubsec{Functions corresponding to each production rule}{parsingfun}
In a recursive descent parser there is a function corresponding to every production rule. These functions
receive the context of parsing (i.e. the text to be parsed and the current position) and return a Boolean
value which is true if the function could match a certain substring of the text and could move the position
further towards the end of the text. If the function returns false, the position is not altered.

Following is an example function that matches one character of the text if it can be found in the string
serving as a character set taken as an argument. The structure of this function is characteristic of the
rule methods in the SyntX library.

\begin{center}
	\begin{minipage}[h]{0.8\textwidth}
		\begin{lstlisting}[language=C++, breaklines=true, numbers=left]
bool character(match_range &context, std::string const &characters) {
  match_range local = context;

  if (local.first == local.second) return false;

  for (auto c: characters) {
    if (*local.first == c) {
      ++local.first;
      context.first = local.first;
      return true;
    }
  }

  return false;
}
		\end{lstlisting}
	\end{minipage}
\end{center}

The \texttt{match\_range} type is an \texttt{std::pair} that holds two \texttt{std::string::iterator}s. It
describes the context of the parsing: the first element points to the current position (the character that
should be analyzed next), the second to the end of the text.

\begin{center}
	\begin{minipage}[h]{0.8\textwidth}\label{lst:matchrange}
		\begin{lstlisting}[language=C++, breaklines=true]
using match_range = std::pair< std::string::const_iterator,
                               std::string::const_iterator >;
		\end{lstlisting}
	\end{minipage}
\end{center}

The function has two arguments, the context and the character set that contains the characters that are
accepted. The context is taken as a non-constant reference because the function sets the current position
after the the last character it could match.

In line~2 a copy of the context is created. This copy will reflect the analysis performed by the function. A
complex parser function can delve deep into a string before finding out that it doesn't match it after all. It
might also call a series of other parsing functions on the way that also alter this value, so it is absolutely
necessary to keep a copy of the original position and only change that value when there is a successful match.

Every parser function that operates at the character level should always check whether the end of the text has
been reached. This can be seen in line~4.

As long as a function finds characters it can consume, it advances the current position, i.e. the first
element in the local copy of the context (line~8).

If a function matches a certain substring of the text and can not advance further, two things need to be done:
the context taken by reference has to be changed to reflect the advancement in the analysis of the text and it
has to return true (lines~9-10). Otherwise it has to merely return false (line~14), the context is left
unchanged.

\usubsec{Recursive descent}{recdesc}
The example above showed a simple a parsing function working at the character level. It doesn't call any other
function, instead it decides on its own whether the text at the given position matches it or not. This is
because that rule describes a so called \emph{terminal symbol}, one that corresponds to a symbol actually
appearing in the text, in this case, a letter.

Other rules might define symbols that contain other symbols and define a structure that these symbols should
have in order to comply to the rule. The composite symbols are called \emph{non-terminal symbols}.

In RDPs non-terminal symbols can be parsed using functions that call other parsing functions just as any rule
can be referenced in the definition of an EBNF rule.

Let's see an example for such a function: the \texttt{expression} rule in the simple grammar seen earlier.

\begin{center}
	\begin{minipage}[h]{0.8\textwidth}
		\begin{lstlisting}[language=C++, breaklines=true, numbers=left]
bool expression(match_range &context) {
	match_range local = context;

	if  (
		character(local, "(") && addition(local) && character(local, ")")
	) {
		context.first = local.first;
		return true;
	}

	return false;
}
		\end{lstlisting}
	\end{minipage}
\end{center}

It is interesting to note that as this function doesn't operate at the character level (every character
consumed by the rule is processed by one of the functions called by it), it doesn't need to check for the end
of the text -- it has to be done by the low-level functions.

The sequence of the sub-rules is realized by the \texttt{\&\&} operator of the C++ language. Short-circuit
evaluation is exploited here: if the first rule doesn't match and thus returns \texttt{false} then the second
is not called and the entire expression will get a \texttt{false} value.

Furthermore the order of evaluation is fixed too and goes from the left to the right. So \texttt{addition}
will receive an updated \texttt{local} -- the value that was altered by the first call to \texttt{character}.

So when all three functions return \texttt{true}, the body of the \texttt{if} statement is evaluated and
\texttt{context} receives a value updated by all three functions and now pointing to the next position of the
text to be parsed.

Alternatives can be realized using the \texttt{||} operator where short-circuit evaluation comes handy again
as the second function gets called only if the first failed to match (in which case the first doesn't alter
the context which is also important).

Please note that composite logical expression should not be constructed as they can lead to mispositioned
iterators. Let's investigate the following code fragment:

\begin{center}
	\begin{minipage}[h]{0.8\textwidth}
		\begin{lstlisting}[language=C++, breaklines=true, numbers=left]
if (
	(rule1(local) && rule2(local)) || rule3(local)
) {
	...
}
		\end{lstlisting}
	\end{minipage}
\end{center}

Let's assume that \texttt{rule1} matches but \texttt{rule2} doesn't, so \texttt{rule3} gets a chance. The
problem is that \texttt{rule1} moves the position that \texttt{local} points to. Unfortunately it doesn't get
corrected before \texttt{local} is fed to \texttt{rule3} so \texttt{rule3} will try to match from a different
position then the one where \texttt{rule1} started from and this is not what we intended to do.

If the above expression is realized in two seperate functions, one containing the sequence (AND logic) and the
other containing the option (OR logic) then this situation doesn't occur as the functions will only alter the
context if they match. This is done automatically in the SyntX framework but it is something the programmer
has to pay attention to if the parser is handwritten.

\usubsec{Actions during parsing}{actions}
All the example functions shown up to here did was to tell whether their input complied to their requirements
or not. If the task of the parser is merely to determine if a text conforms to a specific grammar then this is
sufficient. Otherwise the functions should perform operations to yield a result of the parsing. This can be in
the form of an AST (abstract syntax tree) or pratically any data that can be generated based on the input.

When the parsing functions are handwritten the actions to be performed when a match is found can be put inside
the parsing function resulting in a code where parsing and data processing is intertwined. For simple problems
this is a perfect solution and is easy to handle.

When problems become more complex this method becomes tiresome or even impracticable: there are cases where
data processing can not be performed at the time of parsing as additional knowledge is needed that can only be
extracted later, possibly after the parsing of the entire input is finished.

In the SyntX framework an \texttt{std::function} (which can wrap a plain function, a method, a function object
or a lambda) can be assigned to any rule -- these are called \emph{action}s. The function receives the matched
range as a \texttt{std::string} and can do whatever is needed when the given rule is successfully matched.

There is just one problem with this approach: if a complex data is defined by a sequence of rules such as in
this case

\begin{center}
	\begin{minipage}[h]{0.8\textwidth}
		\begin{lstlisting}[breaklines=true]
complex = rule1 rule2 rule3
		\end{lstlisting}
	\end{minipage}
\end{center}

then we can not be sure that the entire expression will match until \texttt{rule3} returns \texttt{true} but
we need to extract the results on a rule-basis, otherwise we get the matched range of the three rules packed
in one string needing yet another extraction.

A solution to this problem can be to store the results of the three rules in temporary variables and build the
complex data only when all three matched successfully.

In SyntX actions are added to rules using \texttt{operator[]} and they can be assigned to a group of rules as
well:

\begin{center}
	\begin{minipage}[h]{0.8\textwidth}
		\begin{lstlisting}[breaklines=true]
complex = ( rule1[action1] rule2[action2] rule3[action3] )[action4]
		\end{lstlisting}
	\end{minipage}
\end{center}

Please note that this is not exactly the correct syntax -- it is only shown for demonstration purposes here.

\usec{The SyntX framework}{syntx}
In the next few sections the framework's structure and the basics of its operation are explaing briefly. For
the details please refer to the Doxygen documentation, which can be generated from the source code with the
help of the Makefile provided with the project (\texttt{make docs}).

\usubsec{The \texttt{base\_rule} class}{baserule}
The base class of every rule is \texttt{base\_rule}. It defines two data types used throughout the framework,
contains the semantic action assigned to a rule and performs the basic administrative tasks concering the
matching process. The exact instructions regarding the matching have to go in a virtual function defined as
pure virtual in this class (\texttt{test}).

One of the data types define in \texttt{base\_rule} has already been mentioned on page
\pageref{lst:matchrange}, it is the \texttt{match\_range} which is used for two purposes: it defines the
limits between which the parsing is done and also the range matched by a given rule.

The other type is \texttt{semantic\_action} discussed earlier on page \pageref{subsec:actions}:

\begin{center}
	\begin{minipage}[h]{0.8\textwidth}
		\begin{lstlisting}[language=C++, breaklines=true]
using semantic_action = std::function<void(std::string const &)>;
		\end{lstlisting}
	\end{minipage}
\end{center}

One action can be assigned to a rule and it's stored in the \texttt{the\_semantic\_action} data member of the
class.

The \texttt{match} method handles the tasks associated with matching.

\begin{center}
	\begin{minipage}[h]{0.8\textwidth}
		\begin{lstlisting}[language=C++, breaklines=true, numbers=left]
bool base_rule::match(match_range &context, match_range &the_match_range) {
	match_range a_range;

	if (test(context, a_range)) {
		the_match_range.first = a_range.first;
		the_match_range.second = a_range.second;

		if (the_semantic_action) {
			std::string the_matched_substring(the_match_range.first, the_match_range.second);
			the_semantic_action(the_matched_substring);
		}

		return true;
	}
	else return false;
}
		\end{lstlisting}
	\end{minipage}
\end{center}

The \texttt{match} method calls the virtual \texttt{test} to find out whether the text conforms to the rule at
the current position. If it doesn't, the method simply returns \texttt{false}, while in the case of a
successful match, the matched range is stored in the local variable \texttt{a\_range}. This value is delegated
to \texttt{the\_match\_range}, which is a reference of a variable taken as an argument.

If a semantic action has been assigned to the rule, it is called and the result of the matching process is
given to it as a \texttt{std::string}.

The \texttt{base\_rule} class has a virtual funtion called \texttt{clone} the purpose of which is to make the
entire class hierarchy clonable. More on this can be found on page \pageref{subsec:rule}.

\usubsec{Example of a rule subclassed from \texttt{base\_rule}}{character}
\usubsec{Realizing EBNF operators}{ebdfoperators}
\usubsec{The \texttt{rule} class}{rule}

\newpage \tableofcontents

\vfill
\vfill
\begin{center}
\footnotesize\emph{Please consider the environment before printing this document.}
\end{center}
\vfill
\end{document}
